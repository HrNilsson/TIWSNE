%!TEX root = TIWSNE_Mini_project_main.tex
\section{Compression}
When compressing an image there are two overall strategies to follow:
\emph{Lossless} or \emph{lossy} compression.

Lossless compression means encoding and repackaging data in order to remove as much redundant information as possible, while maintaining all information needed for a perfect reconstruction. 
One method of doing this is called \emph{Huffman coding}.
This involves building a code tree or "alphabet" with variable symbol length.
Symbols, or in our case pixel intensities, that occur more often and hence contain less information, are encoded with a shorter bit length. 
This increases the average information per symbol, or entropy, leading to a decrease in file size.

The implementation of Huffman coding is rather processor and memory intensive compared to what is available on the TelosB.
Further, an implementation of Huffman coding is beyond the scope of this course.

Instead, a lossy compression scheme has been implemented in this project.
Lossy compression involves discarding information yielding great decreases in file size at the expense of the ability to perfectly reconstruct the image.
An example of lossy image compression is \emph{JPEG-compression}.
This utilizes Discrete Cosine Transforms, knowledge of human psycho-visual perception and ultimately Huffman Coding to greatly compress images while maintaining a low impact on perceived image quality.
Again, this is more processing and memory intensive than what is feasible on a TelosB mote and beyond the scope of the course.

In this project a simple compression scheme has been implemented, where the two least significant  bits (LSB) is truncated, without regard to the amount of information contained within them.
This effectively reduces the amount of data to $ \frac{3}{4} $ of the original. 

Next is the challenge of compressing the truncated data into a smaller bitstream for transmission.
At this point every 6-bit truncated pixel is still stored in a 8-bit \texttt{nx\_unit8\_t}.
The way this is handled is by utilizing so-called bitfields.


\begin{lstlisting}[
	style		= nesC, 
	captionpos	= b, 
	caption		= Definition of \texttt{imageVector} bitfield from \texttt{imageCompression.h} (See appendix for source code,
	label		= lst:imageVector
	]
typedef nx_struct imageVectors {
	
	// Five pixels in one structure
	nx_uint8_t px0 : 6;
	nx_uint8_t px1 : 6;
	nx_uint8_t px2 : 6;
	nx_uint8_t px3 : 6;
	nx_uint8_t px4 : 6;
	
	// Fill to 32 bits
	nx_uint8_t notFull: 2;
	
} imageVector;

\end{lstlisting}

A bitfield is a custom data type that contains a struct, where the bit length of all the contained variables are customizable, as long as the total amount of bits are consistent with standard integer types.
In this project 32-bit \texttt{nx\_uint32\_t}.

Listing \ref{lst:imageVector} shows the bitfield used to store pixels in this project.
Because our compressed pixelsize of 6 bits does not divide neatly into 32 or 64 bit data types, there is some overhead in the bitfield.
The resulting compression rate is then:

\begin{equation}
R_{compression} = 
\dfrac{32\ bits}
{5 * 8\ bits} =
\dfrac{4}{5} = 
80\ \%
\end{equation}

The extra fill bits are then used for signalling when an end of file is reached and not all fields contain valid pixels.
This is used by the methods that handle compression/decompression. 

To handle compression and decompression, a module \texttt{QuantCompressC} has been implemented that provides an \texttt{interface} (See listing \ref{lst:QC_interface}) with a \texttt{command} for each action.
These are then \texttt{call}ed from within the main state machine (See section \ref{sec:program_flow}).

\begin{lstlisting}[
	style		= nesC, 
	captionpos	= b, 
	caption		= \texttt{Interface} for \texttt{QuantCompressC} (See appendix for source code),
	label		= lst:QC_interface
	]
#include "ImageCompression.h"

interface QuantCompress{
	command imageVector compressVector(nx_uint8_t * inputVec);
	command void deCompressVector(imageVector, nx_uint8_t * outputVec);
}
\end{lstlisting}
