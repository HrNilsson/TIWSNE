%!TEX root = TIWSNE_Mini_project_main.tex
\section{Compression}
When compressing an image there are two overall strategies to follow:
\emph{Lossless} or \emph{lossy} compression.

Lossless compression means encoding and repackaging data in order to remove as much redundant information as possible, while maintaining all information needed for a perfect reconstruction. 
One method of doing this is called \emph{Huffman coding}.
This involves building a code tree or "alphabet" with variable symbol length.
Symbols, or our case pixel intensities, that occur more often and hemce contain less information, are encoded with a shorter bit length. 
This increases that average information per symbol, or entropy, leading to a decrease in file size.

The implementation og Huffman coding is rather processor and memory intensive compared to what is available on the TelosB.
Further, an implementation of Huffman coding is beyond the scope of this course.

In stead, a lossy compression scheme has been implemented in this project.
Lossy compression involves discarding information yielding great decreases in file size at the expense of the ability to perfectly reconstruct the image.
An example of lossy image compressin is \emph{JPEG-compression}.
This utilizes Dicrete Cosine Transforms, knowledge of human psycho-visual perception and ultimately Huffman Coding to greatly compress images while maintaining a low impact on perceived image quality.
Again, this is more processing and memory intensive than what is feasible on a TelosB mote and beyond the scope of the course.

In this project a simple compression scheme has been implemented, where the two least significant  bits (LSB) is truncated, without regard to the amount of information contaied within them.
This effectively reduces the amount of data to $ \frac{3}{4} $ of the original. 

Next is the challenge of compressing the truncated data into a smaller bitstream for transmission.
At this point every 6-bit truncated pixel is still stored in a 8-bit \texttt{unit8\_t}.
The way this is handled is by utilizing so-called bitfields.

\fxnote{Insæt exerpt med bitfield struct}