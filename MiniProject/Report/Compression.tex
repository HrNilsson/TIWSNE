%!TEX root = TIWSNE_Mini_project_main.tex
\section{Compression}
When compressing an image there are two overall strategies to follow:
\emph{Lossless} or \emph{lossy} compression.
Lossless compression means encoding and repackaging data in order to remove as much redundant information as possible. 
One method of doing this is called \emph{Huffman coding}.
This involves building a code tree or "alphabet" with variable symbol length.
Symbols, or our case pixel intensities, that occur more often and hemce contain less information, are encoded with a shorter bit length. 
This increases that average information per symbol, or entropy, leading to a decrease in file size.

The implementation og Huffman coding is rather processor and memory intensive compared to what is available in the.
Further, an implementation of Huffman coding is beyond the scope of this class.

In stead, a lossy compression scheme has been implemented in this project.