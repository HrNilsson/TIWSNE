%!TEX root = TIWSNE_Mini_project_main.tex
\section{Image}

The test image was received correctly on the PC. The result can be seen in figure \ref{fig:compressedlena}.

\begin{figure}[H]
\centering
\begin{subfigure}{.5\textwidth}
  \centering
  \includegraphics[width=.6\linewidth]{lena}
  \caption{The original, uncompressed test image.}
  \label{fig:lena}
\end{subfigure}%
\begin{subfigure}{.5\textwidth}
  \centering
  \includegraphics[width=.6\linewidth]{lenacompressed2bit}
  \caption{The test image after being  compressed - 2 least significant bits truncated.}
  \label{fig:compressedlena}
\end{subfigure}
\caption{The test image with and without compression.}
\label{fig:lenacomp}
\end{figure}

As can be seen on figure \ref{fig:lenacomp}, the image resulting from the simple truncation compression is not visibly different from the original image. Thus the reduction in image quality has not been detrimental.

The correctness of the compression was verified by studying the content of the file before and after compression. A comparison of the two files are shown in figure \ref{fig:hexlenacomp}. 

\begin{figure}[H]
\centering
\begin{subfigure}{.5\textwidth}
  \centering
  \includegraphics[width=0.95\linewidth]{rawhexdump}
  \caption{Hexdump of the original image.}
  \label{fig:lena}
\end{subfigure}%
\begin{subfigure}{.5\textwidth}
  \centering
  \includegraphics[width=0.95\linewidth]{compressedhexdump}
  \caption{Hexdump of the compressed image.}
  \label{fig:compressedlena}
\end{subfigure}
\caption{Content of the image file with and without compression.}
\label{fig:hexlenacomp}
\end{figure}


%Hexdump
%discussion: harder compression